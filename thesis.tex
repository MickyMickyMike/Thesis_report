% options:
% thesis=B bachelor's thesis
% thesis=M master's thesis
% czech thesis in Czech language
% slovak thesis in Slovak language
% english thesis in English language
% hidelinks remove colour boxes around hyperlinks

\documentclass[thesis=M,czech,hidelinks]{FITthesis}[2013/05/06]

\usepackage[utf8]{inputenc} % LaTeX source encoded as UTF-8

\usepackage{pdfpages}

\usepackage{graphicx} %graphics files inclusion
% \usepackage{amsmath} %advanced maths
% \usepackage{amssymb} %additional math symbols

\usepackage{dirtree} %directory tree visualisation

% % list of acronyms
% \usepackage[acronym,nonumberlist,toc,numberedsection=autolabel]{glossaries}
% \iflanguage{czech}{\renewcommand*{\acronymname}{Seznam pou{\v z}it{\' y}ch zkratek}}{}
% \makeglossaries

\newcommand{\tg}{\mathop{\mathrm{tg}}} %cesky tangens
\newcommand{\cotg}{\mathop{\mathrm{cotg}}} %cesky cotangens

\setcounter{tocdepth}{2}
\usepackage{listings}
\usepackage{multirow}
\usepackage{hyperref}
\usepackage{subcaption}
\usepackage{graphicx} 
\usepackage{epstopdf}
\usepackage{minted}
\usepackage{tcolorbox}
\usepackage{etoolbox}
\BeforeBeginEnvironment{minted}{\begin{tcolorbox}}%
\AfterEndEnvironment{minted}{\end{tcolorbox}}%

\usepackage{xcolor}
%\let\oldtexttt\texttt
%\renewcommand{\texttt}[2][blue]{\textcolor{#1}{\ttfamily #2}}

%\usemintedstyle{borland}

% % % % % % % % % % % % % % % % % % % % % % % % % % % % % % 
% ODTUD DAL VSE ZMENTE
% % % % % % % % % % % % % % % % % % % % % % % % % % % % % % 

\department{Katedra řídicí techniky}
\title{Neco jako doplnovani cleansetu blabla a tak}
\authorGN{Michal} %(křestní) jméno (jména) autora
\authorFN{Staněk} %příjmení autora
\authorWithDegrees{Michal Staněk} %jméno autora včetně současných akademických titulů
\supervisor{Ing. Jan Kubr, Ph.D.}
\acknowledgements{Chtěl bych poděkovat panu Ing. Janu Kubrovi, Ph.D. za odborné vedení mé práce, za pomoc a věcné rady při zpracování této práce.}
\abstractCS{TODO}
\abstractEN{TODO}
\placeForDeclarationOfAuthenticity{V~Praze}
\declarationOfAuthenticityOption{4} %volba Prohlášení (číslo 1-6)
\keywordsCS{Python, Virtualbox, Fiddler, html, js}
\keywordsEN{Python, Virtualbox, Fiddler, html, js}
% \website{http://site.example/thesis} %volitelná URL práce, objeví se v tiráži - úplně odstraňte, nemáte-li URL práce

\begin{document}

% \newacronym{CVUT}{{\v C}VUT}{{\v C}esk{\' e} vysok{\' e} u{\v c}en{\' i} technick{\' e} v Praze}
% \newacronym{FIT}{FIT}{Fakulta informa{\v c}n{\' i}ch technologi{\' i}}



\definecolor{mygreen}{rgb}{0,0.6,0}
\definecolor{mygray}{rgb}{0.95,0.95,0.95}
\definecolor{mymauve}{rgb}{0.58,0,0.82}

\lstset{ %
  xleftmargin=10pt,xrightmargin=10pt,
  frame=tbrl,
  framerule=1pt,
  language=Java,
  backgroundcolor=\color{mygray},   % choose the background color  
  commentstyle=\color{mygreen},    % comment style
  escapeinside={\%*}{*)},          % if you want to add LaTeX within your code
  keywordstyle=\color{blue},       % keyword style
  stringstyle=\color{mymauve},     % string literal style
} 
%\setlength{\parskip}{10pt}

\chapter{Úvod}
Dnešní doba je plná rizik, která představují hrozbu pro každodenního uživatele internetu. Ať už se jedná o phishing (zisk citlivých údajů pomocí podvodné internetové komunikace) či různé druhy malwaru (nežádoucí programy mající za úkol poškodit uživatele). V boji s těmito riziky je důležité chránit sebe a svoje data pomocí antivirových programů. Jedním z nejrozšířenějších je Avast, který má přes 435 milionů aktivních uživatelů a měsíčně zabrání okolo 2 miliardám útoků\cite{avast_flier}.

\section{Motivace}
Avast, stejně jako většina antivirových programů, uchovává informace o všech známých škodlivých entitách. Tato databáze se denně rozšiřuje o spousty nových záznamů, které obsahují nejen informace o celých souborech, ale i kusy kódu webového obsahu (tzv. string detekce), které jsou považovány za příznak podvodných úmyslů. Může se však stát, že je tento kus kódu moc obecný a dochází tak i k blokování čistého obsahu (tzv. fals-positive detekcím). Aby se těmto situacím předcházelo, je zapotřebí udržovat i databázi s čistými záznamy (tzv. cleanset). Tyto záznamy jsou převážně HTML a js soubory.

Dříve, než se nová string detekce začlení do jádra antiviru, je její obsah porovnán se všemi záznamy na cleansetu a pokud dojde ke shodě (tj. detekční string je součástí nějakého souboru na cleansetu), je tato detekce považována za nevalidní. Tímto dochází k zabránění fals-positive detekcím. 

Ideálním stavem je tedy mít záznam o veškerém čistém obsahu internetu, což je samozřejmě nemožné. Avšak čím více záznamů cleanset obsahuje, tím kvalitnější je běh antivirového programu. V současné době dochází k doplňování cleansetu pouze občasně a to převážně manuálně za pomoci jednoduchých scriptů.

\section{Cíle práce} 
 Hlavním cílem této práce je vytvořit plně automatizovaný systém, který bude databázi s čistými záznamy periodicky doplňovat o nový obsah, čímž by mělo dojít ke zlepšení funčnosti antivirového programu. Dále bude potřeba systém začlenit do již stávající  infrastruktury. Primárním úkolem je tedy vytvořit projekt, který by modernizoval doplňování cleansetu, avšak současně je možné systém zobecnit k využití i v jiných aplikacích. Celý proces bude řádně otestovaný a bude zhodnocen jeho přínos systému.


\section{Popis systému}
Platforma se skládá z autodráhových vozidel typu Carrera Ford Capri (Obr.\ref{fig:kolona}).
\begin{figure}[h]
        \centering
      %  \includegraphics[width=9cm]{pictures/autodraha.JPG}
        \caption{Kolona vozidel}
        \label{fig:kolona}
\end{figure}

 \begin{figure}[h]
         \centering

         \begin{minipage}[b]{0.49\textwidth}
     %            \includegraphics[width=6cm]{pictures/car2.JPG}
                 \caption*{(a) Pohled po odmontování kapoty}
                 \label{fig:car2}
 		\end{minipage}
         \begin{minipage}[b]{0.49\textwidth}
     %            \includegraphics[width=6cm]{pictures/car3.JPG}
                 \caption*{(b) Spodní STM modul a DC motor}
                 \label{fig:car1}
         \end{minipage}
 	\caption{Vnitřek vozidel}
 	\label{fig:cars}
 \end{figure}

\chapter{Použité technologie}

\section{Python 3.7}
Python je skriptovací programovací jazyk, jehož syntaxe je lehce odlišná od konvenčních programovacích jazyků (Java, C) v tom, že nepoužívá středníky ani složené závorky. Jedná se o hybridní programovací jazyk, což znamená, že program nemusí být nutně objektově orientovaný, ale části mohou mít více procedurální charakter. Tím dochází k lepší čitelnosti kódu a celkovému zjednodušení. Síla Pythonu je i ve velkém množství balíků s knihovnami, které podporují jeho všestrannost. Kvůli těmto vlastnostem byl vybrán pro tuto diplomovou práci.

\section{Fiddler}
Fiddler je nástroj vyvíjen firmou Telerik, sloužící k zachytávání internetové komunikace. Funguje na principu MitM (Man-in-the-middle) útoku, kdy se útočník vtěsná mezi dva účastníky internetového provozu a nechá je komunikovat skrz sebe. Zde je však tento útok chtěný. Jeho automatizace lze docílit inicializačním souborem, který obsahuje různá pravidla a je psaný v javascriptu. Při správném nastavení je fiddler schopný zachytávat i šifrovanou komunikaci, kvůli čemuž byl použit v této práci.

\section{Selenium}
Selenium je opensource nástroj používaný k automatizovanému přístupu k webovým aplikacím. Těchto vlastností se často využívá při testování, avšak v této práci je použit pouze k obsluze webového prohlížeče. Selenium obsahuje vlastní vývojové prostředí, které lze využít bez velké znalosti programování, existují však i jeho implementace do většiny populárních programovacích jazyků.


\section{Jenkins}

\section{Kafka}

\chapter{Analýza a řešení problému}
Problém, kterým se tato práce zabývá, se skládá z vícero dílčích podproblémů. Z bezpečnostních důvodů je kladen důraz na to, aby byl veškerý proces stahování webového obsahu spuštěn na virtuálním stroji. Předpokládá se sice, že všechny stažené soubory budou nezavirované, avšak spoléhat se na to není moc bezpečné řešení. V situaci, kdy by došlo ke stažení nakaženého souboru, lze virtuální stroj jednoduše ukončit a vrátit do původního stavu. Daleko snáz než jiný, reálný systém.

Samotné stahování souborů je další samostatný podproblém. Zde by měla být implementována logika selekce potřebných souborů, protože cleanset obsahuje pouze HTML a js soubory.

Posledním krokem bude přesun získaného obsahu do samotné databáze cleansetu. Jednotlivé podproblémy jsou tedy následující:
\begin{itemize}
	\item Spouštění a obsluha virtuálního stroje
	\item Zisk čistých souborů z webových stránek
	\item Nahrání získaného obsahu do databáze
\end{itemize}

\section{Souštění a obsluha virtuálního stroje}

\section{Zisk čistých souborů z webových stránek}

\section{Nahrání získaného obsahu do databáze}


\begin{eqnarray}
\label{konstanty}
k_p &=& 3 \\ \nonumber
k_i &=& 0.8
\end{eqnarray}

\begin{figure}[h]               
  \begin{minted}{c}
void SISOcontroller_step(void) {
  
  SISOcontroller_Y.Out1 = 3.0 * SISOcontroller_U.In1 
     + SISOcontroller_DW.Integrator_DSTATE;
    
  SISOcontroller_DW.Integrator_DSTATE += 
     0.8 * SISOcontroller_U.In1;
  }
  \end{minted}      
  \caption{Implementace bloku PI regulátor v jazyce C}
  \label{fig:ctrl.c}
\end{figure}

\begin{figure}[h]               
  \begin{minted}{Java}
private native float generatedStep(float e); 
static {
  try{
    System.load("/slotcar/lib/libSISOController.so");
  }
  catch(UnsatisfiedLinkError e){
    System.out.println("Cannot load library.");
  }
} 
  \end{minted}      
  \caption{Deklarace nativní metody a načtení sdílené knihovny}
  \label{fig:SISO.java}
\end{figure}


\chapter{Testování a experimentální ověření}

\chapter{Instrukce k zacházení}


 \setlength{\parskip}{10pt}

\begin{conclusion}

\end{conclusion}

\bibliographystyle{csn690}
\bibliography{mybibliographyfile}
\begin{thebibliography}{9}

    \bibitem{avast_flier}
    \textit{Avast corporate factsheet}, Dostupné z: \\       \url{https://cdn2.hubspot.net/hubfs/2706737/media-materials/corporate-factsheet/Avast_corporate_factsheet_A4_en.pdf} \\
       	
    \bibitem{jan}
       	Moravec Jan. \textit{Distribuované řízení kolon vozidel na autodráze}. \textcopyright2014, České vysoké učení technické v Praze, vedoucí práce Ing. Ivo Herman, Dostupné z: \\    \url{https://dspace.cvut.cz/bitstream/handle/10467/24299/F3-BP-2014-Moravec-Jan-prace.pdf} \\

\end{thebibliography}

\appendix

%\chapter{Seznam použitých zkratek}
% \printglossaries
%\begin{description}
%	\item[GUI] Graphical user interface
%	\item[XML] Extensible markup language
%\end{description}


% % % % % % % % % % % % % % % % % % % % % % % % % % % % 
% % Tuto kapitolu z výsledné práce ODSTRAŇTE.
% % % % % % % % % % % % % % % % % % % % % % % % % % % % 
% 
% \chapter{Návod k~použití této šablony}
% 
% Tento dokument slouží jako základ pro napsání závěrečné práce na Fakultě informačních technologií ČVUT v~Praze.
% 
% \section{Výběr základu}
% 
% Vyberte si šablonu podle druhu práce (bakalářská, diplomová), jazyka (čeština, angličtina) a kódování (ASCII, \mbox{UTF-8}, \mbox{ISO-8859-2} neboli latin2 a nebo \mbox{Windows-1250}). 
% 
% V~české variantě naleznete šablony v~souborech pojmenovaných ve formátu práce\_kódování.tex. Typ může být:
% \begin{description}
% 	\item[BP] bakalářská práce,
% 	\item[DP] diplomová (magisterská) práce.
% \end{description}
% Kódování, ve kterém chcete psát, může být:
% \begin{description}
% 	\item[UTF-8] kódování Unicode,
% 	\item[ISO-8859-2] latin2,
% 	\item[Windows-1250] znaková sada 1250 Windows.
% \end{description}
% V~případě nejistoty ohledně kódování doporučujeme následující postup:
% \begin{enumerate}
% 	\item Otevřete šablony pro kódování UTF-8 v~editoru prostého textu, který chcete pro psaní práce použít -- pokud můžete texty s~diakritikou normálně přečíst, použijte tuto šablonu.
% 	\item V~opačném případě postupujte dále podle toho, jaký operační systém používáte:
% 	\begin{itemize}
% 		\item v~případě Windows použijte šablonu pro kódování \mbox{Windows-1250},
% 		\item jinak zkuste použít šablonu pro kódování \mbox{ISO-8859-2}.
% 	\end{itemize}
% \end{enumerate}
% 
% 
% V~anglické variantě jsou šablony pojmenované podle typu práce, možnosti jsou:
% \begin{description}
% 	\item[bachelors] bakalářská práce,
% 	\item[masters] diplomová (magisterská) práce.
% \end{description}
% 
% \section{Použití šablony}
% 
% Šablona je určena pro zpracování systémem \LaTeXe{}. Text je možné psát v~textovém editoru jako prostý text, lze však také využít specializovaný editor pro \LaTeX{}, např. Kile.
% 
% Pro získání tisknutelného výstupu z~takto vytvořeného souboru použijte příkaz \verb|pdflatex|, kterému předáte cestu k~souboru jako parametr. Vhodný editor pro \LaTeX{} toto udělá za Vás. \verb|pdfcslatex| ani \verb|cslatex| \emph{nebudou} s~těmito šablonami fungovat.
% 
% Více informací o~použití systému \LaTeX{} najdete např. v~\cite{wikilatex}.
% 
% \subsection{Typografie}
% 
% Při psaní dodržujte typografické konvence zvoleného jazyka. České \uv{uvozovky} zapisujte použitím příkazu \verb|\uv|, kterému v~parametru předáte text, jenž má být v~uvozovkách. Anglické otevírací uvozovky se v~\LaTeX{}u zadávají jako dva zpětné apostrofy, uzavírací uvozovky jako dva apostrofy. Často chybně uváděný symbol "{} (palce) nemá s~uvozovkami nic společného.
% 
% Dále je třeba zabránit zalomení řádky mezi některými slovy, v~češtině např. za jednopísmennými předložkami a spojkami (vyjma \uv{a}). To docílíte vložením pružné nezalomitelné mezery -- znakem \texttt{\textasciitilde}. V~tomto případě to není třeba dělat ručně, lze použít program \verb|vlna|.
% 
% Více o~typografii viz \cite{kobltypo}.
% 
% \subsection{Obrázky}
% 
% Pro umožnění vkládání obrázků je vhodné použít balíček \verb|graphicx|, samotné vložení se provede příkazem \verb|\includegraphics|. Takto je možné vkládat obrázky ve formátu PDF, PNG a JPEG jestliže používáte pdf\LaTeX{} nebo ve formátu EPS jestliže používáte \LaTeX{}. Doporučujeme preferovat vektorové obrázky před rastrovými (vyjma fotografií).
% 
% \subsubsection{Získání vhodného formátu}
% 
% Pro získání vektorových formátů PDF nebo EPS z~jiných lze použít některý z~vektorových grafických editorů. Pro převod rastrového obrázku na vektorový lze použít rasterizaci, kterou mnohé editory zvládají (např. Inkscape). Pro konverze lze použít též nástroje pro dávkové zpracování běžně dodávané s~\LaTeX{}em, např. \verb|epstopdf|.
% 
% \subsubsection{Plovoucí prostředí}
% 
% Příkazem \verb|\includegraphics| lze obrázky vkládat přímo, doporučujeme však použít plovoucí prostředí, konkrétně \verb|figure|. Například obrázek \ref{fig:float} byl vložen tímto způsobem. Vůbec přitom nevadí, když je obrázek umístěn jinde, než bylo původně zamýšleno -- je tomu tak hlavně kvůli dodržení typografických konvencí. Namísto vynucování konkrétní pozice obrázku doporučujeme používat odkazování z~textu (dvojice příkazů \verb|\label| a \verb|\ref|).
% 
% \begin{figure}\centering
% 	\includegraphics[width=0.5\textwidth, angle=30]{cvut-logo-bw}
% 	\caption[Příklad obrázku]{Ukázkový obrázek v~plovoucím prostředí}\label{fig:float}
% \end{figure}
% 
% \subsubsection{Verze obrázků}
% 
% % Gnuplot BW i barevně
% Může se hodit mít více verzí stejného obrázku, např. pro barevný či černobílý tisk a nebo pro prezentaci. S~pomocí některých nástrojů na generování grafiky je to snadné.
% 
% Máte-li například graf vytvořený v programu Gnuplot, můžete jeho černobílou variantu (viz obr. \ref{fig:gnuplot-bw}) vytvořit parametrem \verb|monochrome dashed| příkazu \verb|set term|. Barevnou variantu (viz obr. \ref{fig:gnuplot-col}) vhodnou na prezentace lze vytvořit parametrem \verb|colour solid|.
% 
% \begin{figure}\centering
% 	\includegraphics{gnuplot-bw}
% 	\caption{Černobílá varianta obrázku generovaného programem Gnuplot}\label{fig:gnuplot-bw}
% \end{figure}
% 
% \begin{figure}\centering
% 	\includegraphics{gnuplot-col}
% 	\caption{Barevná varianta obrázku generovaného programem Gnuplot}\label{fig:gnuplot-col}
% \end{figure}
% 
% 
% \subsection{Tabulky}
% 
% Tabulky lze zadávat různě, např. v~prostředí \verb|tabular|, avšak pro jejich vkládání platí to samé, co pro obrázky -- použijte plovoucí prostředí, v~tomto případě \verb|table|. Například tabulka \ref{tab:matematika} byla vložena tímto způsobem.
% 
% \begin{table}\centering
% 	\caption[Příklad tabulky]{Zadávání matematiky}\label{tab:matematika}
% 	\begin{tabular}{|l|l|c|c|}\hline
% 		Typ		& Prostředí		& \LaTeX{}ovská zkratka	& \TeX{}ovská zkratka	\tabularnewline \hline \hline
% 		Text		& \verb|math|		& \verb|\(...\)|	& \verb|$...$|		\tabularnewline \hline
% 		Displayed	& \verb|displaymath|	& \verb|\[...\]|	& \verb|$$...$$|	\tabularnewline \hline
% 	\end{tabular}
% \end{table}
% 
% % % % % % % % % % % % % % % % % % % % % % % % % % % % 

\chapter{Obsah přiloženého CD}

%upravte podle skutecnosti

\begin{figure}
	\dirtree{%
		.1 slotcar-sw\DTcomment{adresář s Java projektem}.
		.1 SimulinkControllers.
		.2 SISO\DTcomment{jednovstupový regulátor}.
		.2 TwoInputSingleOutput\DTcomment{dvouvstupový regulátor}.
		.2 MISO\DTcomment{vícevstupový regulátor}.
		.2 transferscript.bat\DTcomment{skript pro přenos souborů}.
		.1 text.
		.2 thesis.pdf\DTcomment{text práce ve formátu PDF}.
		.2 thesis.tex\DTcomment{text práce ve formátu \LaTeX}.
		.2 pictures\DTcomment{zdrojové obrázky pro formát \LaTeX}.
		.1 video.
		.2 tutorial.mp4\DTcomment{instruktážní video}.
	}
\end{figure}

\end{document}
